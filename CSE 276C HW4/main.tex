\documentclass[12pt]{article}
\usepackage{graphicx}

\begin{document}
\centerline{\bf Homework 4 - CSE 276C - Math for Robotics}
\centerline{Due: 21 November 2021}

\begin{enumerate}
\item In robotics it is typical to have to recognize objects in the environment.
  We will here use the German Traffic Sign dataset for recognition of traffic signs. You can download the dataset from the link below.

  To reduce computational time, please use the file Train\_subset.csv to read in the train set. Similarly, please use the file Test\_subset.csv to read in the test set.

https://www.kaggle.com/meowmeowmeowmeowmeow/gtsrb-german-traffic-sign

   Compute subspaces for the PCA and LDA methods. Provide
  illustration of the respective 1st and 2nd eigenvectors.
  

  Compute the recognition rates for the test set. Report
  \begin{itemize}
  \item Correct classification
  \item Incorrect classification
  \end{itemize}

  Provide at least one suggestion for how you might improve performance of each method.

\item Consider a predator-prey dynamics such as the simple
  Lotka-Volterra model
  \begin{eqnarray*}
    \mathbf{x}' &=& \mathbf{f}(\mathbf{x})\\
    \mathbf{x} &=& \left( \begin{array}{c}  x_1 \\ x_ 2 \end{array}
    \right) = \left( \begin{array}{c} Prey~polution \\ Predator~population \end{array}\right)\\
    \mathbf{f}(\mathbf{x}) &=& \left( \begin{array}{c} (b - p x_2) x_1 \\
                                        (r x_1 - d) x_2\end{array}\right)
  \end{eqnarray*}
  Without predators, the prey population increases (exponentially)
  without bound, whereas without prey, the predator population
  diminishes (exponentially) to zero. The nonlinear interaction, with
  predators eating prey, tends to diminish the prey population and
  increase the predator population. Use your Runga-Kutta to solve this
  system, with the values b = p = r = d = 1, $x_1(0)$ = 0.3, and $x_2(0)$ =
  0.2.
  
\end{enumerate}
For all questions provide a description of the approach adopted, the
associated code and a description of your results. 


%%% Local Variables:
%%% mode: latex
%%% TeX-master: t
%%% End:

\end{document}

%%% Local Variables:
%%% mode: latex
%%% TeX-master: t
%%% End:
